 % PrivateMsg
  The preceding sections established the relationship between liquidity constraints, consumption concavity, and prudence. This section derives the last step to understand the relationship between liquidity constraints and precautionary saving. First, we explain how prudence of the value function affects precautionary saving. Theorem \ref{thm:riskandconstraints} then shows how the introduction of an additional constraint induces agents to increase precautionary saving when they face a \textit{current} risk. The results in Theorem \ref{thm:riskandconstraints} cannot be generalized to an added risk or liquidity constraint in a later time period because it may hide or alter the effects of current constraints or risks and thereby affect local precautionary saving. The main conceptual issue with having both risks and constraints is that the trick with the \textit{relevant} constraints applied in Section \ref{sec:LCandCC} no longer applies in a setting with risk because constraints may be relevant for some sample paths. However, we still derive our most general result in Theorem \ref{thm:CCandPS}: the introduction of an additional risk results in more precautionary saving in the presence of \textit{all} future risks and constraints than in the case with no future risks and constraints. %In other words (Corollary \ref{cor:CCRISK}), the consumption function is concave in the presence of all future risks and constraints.

  %% Old: The main step in the proof was to be careful in defining an ordered set of relevant constraints. We are now going to be explicit about the effects of current risk on precautionary saving in the presence of all future risks and constraints.


